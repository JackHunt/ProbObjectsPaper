% $Id: template.tex 11 2007-04-03 22:25:53Z jpeltier $

\documentclass{vgtc}                          % final (conference style)
%\documentclass[review]{vgtc}                 % review
%\documentclass[widereview]{vgtc}             % wide-spaced review
%\documentclass[preprint]{vgtc}               % preprint
%\documentclass[electronic]{vgtc}             % electronic version

%% Uncomment one of the lines above depending on where your paper is
%% in the conference process. ``review'' and ``widereview'' are for review
%% submission, ``preprint'' is for pre-publication, and the final version
%% doesn't use a specific qualifier. Further, ``electronic'' includes
%% hyperreferences for more convenient online viewing.

%% Please use one of the ``review'' options in combination with the
%% assigned online id (see below) ONLY if your paper uses a double blind
%% review process. Some conferences, like IEEE Vis and InfoVis, have NOT
%% in the past.

%% Figures should be in CMYK or Grey scale format, otherwise, colour 
%% shifting may occur during the printing process.

%% These three lines bring in essential packages: ``mathptmx'' for Type 1 
%% typefaces, ``graphicx'' for inclusion of EPS figures. and ``times''
%% for proper handling of the times font family.

%\usepackage{mathptmx}
\usepackage{color}
\usepackage{graphicx}
\usepackage{times}
\usepackage{amsmath}
\usepackage{amssymb}
\usepackage{bbm}

%% We encourage the use of mathptmx for consistent usage of times font
%% throughout the proceedings. However, if you encounter conflicts
%% with other math-related packages, you may want to disable it.

%% If you are submitting a paper to a conference for review with a double
%% blind reviewing process, please replace the value ``0'' below with your
%% OnlineID. Otherwise, you may safely leave it at ``0''.
\onlineid{0}

%% declare the category of your paper, only shown in review mode
\vgtccategory{Research}

%% allow for this line if you want the electronic option to work properly
\vgtcinsertpkg

%% In preprint mode you may define your own headline.
%\preprinttext{To appear in an IEEE VGTC sponsored conference.}

%% Paper title.

\title{Title}

%% This is how authors are specified in the conference style

%% Author and Affiliation (single author).
%%\author{Roy G. Biv\thanks{e-mail: roy.g.biv@aol.com}}
%%\affiliation{\scriptsize Allied Widgets Research}

%% Author and Affiliation (multiple authors with single affiliations).
%%\author{Roy G. Biv\thanks{e-mail: roy.g.biv@aol.com} %
%%\and Ed Grimley\thanks{e-mail:ed.grimley@aol.com} %
%%\and Martha Stewart\thanks{e-mail:martha.stewart@marthastewart.com}}
%%\affiliation{\scriptsize Martha Stewart Enterprises \\ Microsoft Research}

%% Author and Affiliation (multiple authors with multiple affiliations)
\author{Author 1\thanks{email@domain}\\ %
        \scriptsize Affiliation %
\and Author 2\thanks{email@domain}\\ %
     \scriptsize Affiliation %
\and Author 3\thanks{email@domain}\\ %
     \parbox{1.4in}{\scriptsize \centering Affiliation}}

%% A teaser figure can be included as follows, but is not recommended since
%% the space is now taken up by a full width abstract.
%\teaser{
%  \includegraphics[width=1.5in]{sample.eps}
%  \caption{Lookit! Lookit!}
%}

%% Abstract section.
\abstract{
	In recent years, major advances have been made in 3D scene reconstruction, with a number of approaches now able to yield dense, globally-consistent models at scale. However, much less progress has been made for objects, which can exhibit far fewer unambiguous geometric/texture cues than a full scene, and thus be much harder to track against.
In this work we present a novel probabilistic object reconstruction framework that simultaneously allows for online, implicit deformation of the objects surface to reduce tracking drift and handle 
loop closure events. Coupled with our probabilistic formulation is the use of a multi subsegment representation of the object, used to enforce global global consistency, with segmentation of the object 
built in to the formulation. Finally, we employ a CRF framework to refine the overall segmentation, defined by a probability field over the object. We present compelling results over the current state-of-the-art 
object reconstruction work and demonstrate robustness and consistency w.r.t. established dense SLAM frameworks.
} % end of abstract

%% ACM Computing Classification System (CCS). 
%% See <http://www.acm.org/class/1998/> for details.
%% The ``\CCScat'' command takes four arguments.

\CCScatlist{ 
  \CCScat{K.6.1}{Cat1}%
{Cat2}{Cat3};
  \CCScat{K.7.m}{Cat1}{Cat2}{Cat3}
}

%% Copyright space is enabled by default as required by guidelines.
%% It is disabled by the 'review' option or via the following command:
% \nocopyrightspace

%%%%%%%%%%%%%%%%%%%%%%%%%%%%%%%%%%%%%%%%%%%%%%%%%%%%%%%%%%%%%%%%
%%%%%%%%%%%%%%%%%%%%%% START OF THE PAPER %%%%%%%%%%%%%%%%%%%%%%
%%%%%%%%%%%%%%%%%%%%%%%%%%%%%%%%%%%%%%%%%%%%%%%%%%%%%%%%%%%%%%%%%
\newcommand{\STUART}[1]{\textcolor{blue}{{\textbf{[Stuart: #1]}}}}
%\newcommand{\STUART}[1]{}

\begin{document}

%% The ``\maketitle'' command must be the first command after the
%% ``\begin{document}'' command. It prepares and prints the title block.

%% the only exception to this rule is the \firstsection command
\firstsection{Introduction}

\maketitle

%% \section{Introduction} 
Dense SLAM (Simultaneous Localisation and Mapping) has proven to be an effective paradigm for the reconstruction of moderately-sized scenes,
with much recent research on the topic driven by the availability of inexpensive, consumer-grade depth sensing equipment \cite{Newcombe2011,Niessner2013,Prisacariu2014}. 
However, a prominent problem within the dense SLAM paradigm is that of tracking camera motion with respect to the scene. \STUART{It's arguable that it's half the point of SLAM, since SLAM involves localisation and mapping, and that's the localisation bit. I would word this slightly differently: it currently sounds a bit self-evident.} Two common approaches exploit geometry and texture 
respectively. \STUART{I know what you mean, but I wouldn't word it like this. It's common for tracking approaches to exploit geometry and/or texture cues, and various approaches exploit one or both to a greater or lesser extent. However, the way you've worded it sounds like you're talking about two specific approaches, neither of which you cite.} However, there are failure cases in both approaches when there is a lack of geometrical or texture based features to which the system must track against. \STUART{Or more precisely, trackers that rely purely on geometry cues tend to fail when there's no geometry, and trackers that rely purely on texture cues tend to fail when there's no texture. As worded, it's a little vague. I think fixing the previous point will help -- you need to make it clear that you're not talking about two specific approaches.}
Common issues range from tracking drift, often caused by ambiguity, to complete tracker failure. \STUART{It's not clear precisely what you mean by `ambiguity' here: this needs pinning down. Also, there are various causes of tracking drift beyond what I think you mean. As a separate question: when you talk about `common issues', what other issues did you have in mind?} In addition, the error incurred during loop closure events can be detrimental 
to tracking quality with the aforementioned approaches. \STUART{It's not obvious to me what you mean by `the aforementioned approaches' here: we're still talking in vague terms about trackers at this point. I think if we're going to point to things that are problematic in existing approaches, we need to make it clear which existing approaches we're talking about.} Failure cases for such approaches are only exacerbated when there is 
less data to track with/against, as is the case with the tracking of an object's pose w.r.t.\ the camera (the inverse of the camera tracking problem).

As object reconstruction can be seen as a smaller-scale equivalent of the scene-based dense reconstruction problem \STUART{Well, \emph{dense} object reconstruction can be.}, it too is prone to the
tracking drift and loop closure problem, sometimes to a prohibitive level. As previously highlighted, the tracking of an object's pose w.r.t.\ the camera is also affected 
by geometric or texture-based ambiguities, caused by a much more focused domain of the data available to the tracking algorithm. \STUART{Are the ambiguities caused by having less data available? Or are there just fewer unambiguous points available for tracking?} For example, when reconstructing an 
entire scene, all of the data in the current and previous frames is available. When tracking solely an object this is not the case. \STUART{Well, the data is arguably available, but you only want to use the bits corresponding to the object for tracking.}

In this paper, we present a probabilistic object reconstruction framework for in-hand reconstruction of rigid objects based on object
appearances. \STUART{Are we doing in-hand? Or are we doing things like chairs? Does our in-hand reconstruction work?} The framework facilitates the correction of camera tracking drift by representing the object to be reconstructed as a
collection of overlapping subsegments such that deformations may be inferred to keep the subsegments aligned, resulting in a consistent
overall model. We utilise a volumetric representation for each of these object subsegments, as with many larger-scale reconstruction
systems\cite{Kahler2016}. \STUART{Are there many systems that do what \cite{Kahler2016} does? Or is there just this one, and we're using it?} Each voxel in the subsegments has additional appearance posterior information pertaining to the voxel's membership of the object.
Over time, multiple volumes containing both surface and probabilistic appearance information are maintained and manipulated to yield a
robust and temporally-consistent model.

In addition, to further increase the robustness of the object tracking, a novel tracking procedure is used, utilising appearance features
in image space. Finally, we optimise for the optimum object shape within a max-flow framework.

We demonstrate an improvement in object reconstruction quality by performing online transformations to the aforementioned subsegments to counteract inconsistencies 
caused by drift in the object pose tracking process. \STUART{An improvement in quality with respect to what?}
We perform (...) experiments and demonstrate a significant qualitative improvement. In addition we demonstrate that the proposed system is quantitatively better 
with regards to (...) by performing (...) experiments.

TO-DO:- Decide upon and perform experiments and expand on this here.

\section{Background and Pertinent Literature}
\subsection{Dense 3D Reconstruction}
Much 3D reconstruction work in recent years has been influenced by the seminal KinectFusion\cite{Newcombe2011} of Newcombe et al in which
RGBD data was integrated into a volumetric representation of the scene, performing simultaneous tracking and mapping. The end result was
high quality 3D static scene models. 
However KinectFusion notably suffered from tracking drift and had no capacity to handle loop closure
events.

Following KinectFusion, Nie{\ss}ner et al present another volumetric reconstruction system\cite{Niessner2013} based around the notion of
hashing regions of space in to voxel blocks. The primary contribution is the ability to scale the abilities of KinectFusion to larger
scenes. However, the contributions do not extend to the camera tracking drift and loop closure problems.

Prisacariu et al present an alternative voxel hashing based system\cite{Prisacariu2014} which provides many optimisations and an open source 
implementation. However, the limitations of\cite{Newcombe2011,Niessner2013} with regards to loop closure and tracking drift are still present. 
However, a later publication\cite{Kahler2016} from the authors of\cite{Prisacariu2014} presents a loop closure and tracking drift reduction 
solution based on the splitting of the scene in to sub scenes with pose adjustments made to tracking constraints between them, with 
a pose graph optimisation as a final processing step. This approach is extremely pertinent to this work as it inspired the multi subsection approach 
that we take. Dai et al recently introduced a system that improves pose estimation for large scale scenes by considering each previously seen frame within a hierarchical framework and perform sparse feature matching to optimise for the camera pose. However, mismatches between keypoints are reported to have an impact on the robustness of the optimisation procedure.

All of the aforementioned voxel based reconstruction works are closely related to and derived from the early work of Curless et al \cite{Curless1996} 
which introduces the notion of encoding isosurfaces as the zero level of a level set function.

\subsection{Object Reconstruction}
In addition to the larger scale dense SLAM works discussed in the previous section, there has been much work on object reconstruction and 
object-centric SLAM. Ren et al \cite{Ren2013} present a probabilistic object tracking and reconstruction system that, like our work, builds 
object reconstructions based on an appearance model. The presented system evolves a level set object representation for voxels that 
are on the object, as per the appearance model. However, the presented system does not have any provision for loop closure detection and 
is prone to tracking drift over time.

Kolev et al present a probabilistic 3D segmentation and surface extraction algorithm\cite{Kolev2006} based on a variational evolution of a level 
set representation. Object appearance probabilities are fused in to the objects volume for the segmentation, such that the algorithm is robust to 
outliers in the observation images. However, their system does not provide any handling of 
loop closure occurrences and the paper makes no reference to tracking integrity. In addition, the images on which their algorithm was tested 
contained only the objects to be reconstructed, with no background noise.

Another volumetric object reconstruction system is presented by Gupta et al\cite{Gupta2016}, using monocular multi view cues. The authors 
perform object segmentation within a graph cut framework to yield object models and perform tracking based on visual and textual cues. 
However the authors report fluctuating camera tracking quality due to the breakage of brightness constancy and specular surfaces. In addition, 
as with all other works introduced up to this point there is no loop closure ability and tracking drift is an issue.

Slavcheva et al present a volumetric object reconstruction system for RGBD sensors in which pose estimates are yielded by registering pairs of 
SDF(Signed Distance Function) volumes. Unlike many of the aforementioned works, the authors do present a loop closure step, however it is 
performed offline as a post processing step. In addition, the tracking procedure of the presented work is reliant on the use of fiducial markers.

Finally, Weise et al present an explicit, surfel based object reconstruction system\cite{Weise2009} based on objects rotating in front of a 3D 
range scanner. During reconstruction an object topology graph is constructed that is used on-line to handle loop closure cases. When a loop 
closure is detected, if there are discrepancies in the topology graph then deformations are applied locally to patches of the object surface. As 
such, the two misaligned ends of the surface are realigned. In addition, the use of the explicit Surfel based representation makes the reconstructed models less amenable to structured  computation(such as processing by a Convolution Neural Network) than their volumetric counterparts. 

\section{Algorithmic Details}
\subsection{Preliminaries}
\subsubsection{Volumetric Representation}
The system we present follows the familiar KinectFusion \cite{Newcombe2011} pipeline for the integration of surface information. In such a 
formulation the scene or object is represented as the zero level of a zero level set \cite{Curless1996}, where the level set is a field of distances 
to the surface and as described, the surface is given by
\begin{equation}
\begin{split}
S = \left\{\psi | \mathcal{D}(\psi) = 0\right\}
\end{split}
\end{equation}
where $S$ is the set of surface voxels, $\psi$ is a voxel in the SDF(Signed Distance Function) volume and $\mathcal{D}(.)$ is the SDF value. 
The SDF volume is truncated to yield a TSDF(Truncated Signed Distance Function) with truncation region $\mu$.

\subsubsection{Pose Estimation}
The pose estimation method used in this work is based on the ICP algorithm as used in \cite{Newcombe2011, Prisacariu2014}. The estimation of 
the camera or object 6-DoF pose is formulated as a minimisation problem of the following form
\begin{equation}
\begin{split}
E = \sum_{\omega \in \mathbf{\Omega}_{d}} \bigg( \big( \mathbf{R}\mathbf{x}_{\omega} + \mathbf{t} - \mathcal{V}(\bar{\omega}) \big)\mathcal{N}(\bar{\omega}) \bigg)^{2}
\end{split}
\end{equation}
where $\mathbf{\Omega}_{d}$ is a depth image, $\omega$ is a 3D point extracted form the depth image, $\mathbf{R}$ is an  $\mathbbm{SO}(3)$ 
rotation matrix, $\mathbf{t}$ is a translation vector, $\mathcal{V}$ is a rendered depth map from the SDF model, $\mathcal{N}$ is a rendered 
normal map from the SDF model and $\bar{\omega}$ is the point $\omega$ projected into the coordinate frame of $\mathcal{V}$ and 
$\mathcal{N}$.

\subsection{Surface Integration}
As in previous works \cite{Newcombe2011,Prisacariu2014} we utilise a weighted mean to fuse new depth measurements in to the TSDF model. As such, 
for a new depth measurement $\eta$ projected to by voxel $\psi$, the following update to the TSDF volume $\mathcal{D}$ is made
\begin{equation}
\begin{split}
\mathcal{D}'(\psi) \leftarrow \frac{w(\psi)\mathcal{D}(\psi) + \min(1, \eta/\mu)}{w(\psi) + 1}
\end{split}
\end{equation}
where $w(.)$ is a weighting function and $\mu$ is the aforementioned TSDF truncation region.

\subsection{Representation and Fusion Procedure}
The representation of the object to be reconstructed makes use of multiple `subvolumes' each pertaining to some patch on the object surface. 
New subvolumes are started when a sufficient amount of new voxels have been allocated and have had data integrated. By ensuring overlap 
between the subvolumes, transformations between them can be found. Using the multiple volume approach allows for pose estimation in each 
volume such that pose estimation discrepancies between subvolumes can be detected and are indicative of pose estimation drift.

The proposed system is inspired by\cite{Kolev2006} in that the representation used for the shape of the object to be modelled is a set of volumes 
of probabilities and surface information. The probabilities are posteriors over a per voxel assignment to either the object voxel set or to the non 
object voxel set. In the proposed system this volume of posterior probabilities is built into with each frame, parallel to the fusion process in systems 
such as KinectFusion\cite{Newcombe2011} and InfiniTAM\cite{Prisacariu2014}.

At each frame a probability map is constructed based on the predictions of the model given the current frame. During the fusion process, this 
smaller volume is mapped in to as a source of voxel wise appearance probability information. The overall appearance based posterior for a given voxel 
$\psi \in \mathbf{\Psi}$ takes the following form:
\begin{equation}
\begin{split}
P(\psi \in \mathbf{\Phi} | \mathbf{\Omega}, \mathbf{p}) = \prod_{t=0}^{\infty} P(\psi_{t} \in \mathbf{\Phi}_{t} | \mathbf{\Omega}_{t}, \mathbf{p}_{t})
\end{split}
\end{equation}
where $\mathbf{\Psi}$ is the volume of voxels for which measurements are accumulated, $\mathbf{\Phi}$ 
is the volume of voxels pertaining to the object, $\mathbf{\Omega}_{t}$ is the current image observation at time $t$ and $\mathbf{p}_{t}$ is the 
currently tracked pose at time $t$.
The above encodes the probability of a voxel belonging to an object as the product of the instantaneous conditionals for observations at each time step. 
Note that in general $\mathbf{\Phi} \subset \mathbf{\Psi}$, and the use of $\mathbf{\Phi}$ in the above equation is an abuse of notation as in the above 
$\mathbf{\Phi}$ is a discretisation of the continuous $\mathbf{\Phi}$ in the probabilistic formulation that follows. Finally, note that a conditional 
independence assumption is made to aid computational tractability in the model.

\subsection{Probabilistic Formulation of Object Reconstruction}
As previously highlighted, central to the proposed system is a volume of posterior probabilities pertaining to a voxel wise membership of either the 
object set or the non object set. This allows one to formulate the full joint distribution over the object as the Probabilistic 
Graphical Model of Figure \ref{pgm1}.
\begin{figure}[h]
	\centering
	\includegraphics{graphical_models/pgm1.pdf}
	\caption{Probabilistic formulation of the object reconstruction pipeline.}
	\label{pgm1}
\end{figure}

Where $\mathbf{\Phi}$ is the shape to be reconstructed, $\mathbf{u}$ is the appearance model volume, $\mathbf{L}$ is the 
set of consistency constraints for each adjacent sub volume pair in the form of rigid transformations, $\mathbf{\Omega}$ is the set of 
RGBD image pixels and $\mathbf{p}$ the set of poses over time.

This gives rise to the following analytical formulation of the above distribution:

\begin{equation}
\begin{split}
P(\mathbf{\Phi}, \mathbf{\Omega}, \mathbf{p}, \mathbf{u}, \mathbf{L}) = 
\prod_{\psi \in \mathbf{\Psi}}\prod_{(s, s') \in \mathcal{S}}P(\mathbf{\Phi}|\mathbf{u}_{v}, \mathbf{L}_{(s, s')}) 
\prod_{t=0}^{\infty}\prod_{p \in \mathcal{P}}\\
P(\mathbf{u_{v}}|\mathbf{\Omega}_{p, t}, \mathbf{p}_{t})
P(\mathbf{L}_{(s, s')}|\mathbf{\Omega}_{p, t}, \mathbf{p}_{t})
P(\mathbf{L}_{(s, s')})P(\mathbf{p}_{t})P(\mathbf{\Omega}_{p, t})
\end{split}
\end{equation}
where $\mathcal{V}$ is the set of voxels across all sub volumes, $\mathcal{P}$ is the set of RGBD pixels for a given 
frame and $\mathcal{S}$ is the set of sub volumes.

However, if one were to assume temporal and pixel wise independence in the RGBD observations and temporal independence in 
the poses, the plate containing $\mathbf{\Omega}$ and $\mathbf{p}$ can be removed:
\begin{equation}
\begin{split}
P(\mathbf{\Phi}, \mathbf{\Omega}, \mathbf{p}, \mathbf{u}, \mathbf{L}) = 
\prod_{v \in \mathcal{V}}P(\mathbf{\Phi}|\mathbf{u}_{v})
\prod_{(s, s') \in \mathcal{S}}P(\mathbf{u_{v}}|\mathbf{\Omega}, \mathbf{p}, \mathbf{L}_{(s, s')})\\
P(\mathbf{L}_{(s, s')}|\mathbf{\Omega}, \mathbf{p}) P(\mathbf{L}_{(s, s')})P(\mathbf{p})P(\mathbf{\Omega})
\end{split}
\end{equation}
In practice this temporal independence assumption causes no issues.

Furthermore, if one assumes voxel wise independence, the plate over voxels can be removed. Finally, assuming $P(\mathbf{p})$ and 
$P(\mathbf{\Omega})$ are uniform distributions, then we have the simpler distribution given by Figure \ref{pgm2}.
\begin{figure}[h]
	\centering
	\includegraphics{graphical_models/pgm2.pdf}
	\caption{Simplified probabilistic formulation of the object reconstruction pipeline.}
	\label{pgm2}
\end{figure}

Where the simpler distribution takes the following analytical form:
\begin{equation}
\begin{split}
P(\mathbf{\Phi}, \mathbf{\Omega}, \mathbf{p}, \mathbf{u}, \mathbf{L}) = 
\prod_{(s, s') \in \mathcal{S}} P(\mathbf{\Phi}|\mathbf{u}, \mathbf{L}_{(s, s')})
P(\mathbf{u}|\mathbf{\Omega}, \mathbf{p})\\
P(\mathbf{L}_{(s, s')}|\mathbf{\Omega}, \mathbf{p})
P(\mathbf{L}_{(s, s')})
\end{split}
\end{equation}

The above formalisms describe a probabilistic framework in which online corrections can be made to the reconstructed model to counteract 
errors incurred by pose tracking inconsistencies. As with previous dense SLAM systems \cite{Newcombe2011, Prisacariu2014, Niessner2013}, 
our system follows a pipeline that consists of a tracking stage and an integration stage. However, our formulation of this pipeline 
consists of an additional estimation module that relies on the use of a subvolume representation to correct tracking errors by applying 
transformations to the subsegments to align them when there are intra subsegment tracking inconsistencies. 
As previously described, during reconstruction the object is split in to subsegments, also referred to as subvolumes, 
with the pose estimation performed in each of the active, visible subsegments. The pose estimation stage for each of these subsegments follows 
the standard ICP(Iterated Closest Point) approach.
As inference on the joint distribution of our model is intractable, conditional independence assumptions are made that do not appear 
to cause any functional issues. The estimation phase of the pipeline is described in the following section.

\subsection{Estimating Deformations}
The tracking consistency constraint denoted by the variable $\mathbf{L}$ in the graphical model given by Figure \ref{pgm1} and Figure \ref{pgm2} can 
be enforced in terms of minimising the transformations between adjacent submaps, such that the camera poses tracked in each subvolume are consistent.  
%The approach proposed in this work differs from \cite{Kahler2016} in that the optimisation is integrated in to the probabilistic formulation previously outlined. 
Given instantaneously inferred transforms between subvolumes obtained from tracking results, 
the objective is to infer a robust, consistent deformation transformation for the subvolume pair.

As such, for each pair of visible subvolumes $(s, s')$, the following posterior must be maximised:
\begin{equation}
\begin{split}
%Bayes rule + chain rule for P(omega, p)
P(\mathbf{\Omega}, \mathbf{p} | \mathbf{L}_{(s, s')}) = \frac{P(\mathbf{L}_{(s, s')} | \mathbf{\Omega}, \mathbf{p}) P(\mathbf{\Omega} | \mathbf{p})P(\mathbf{p})}
{P(\mathbf{L}_{(s, s')})}
\end{split}
\end{equation}
The intuition behind the above equation is that the deformation $\mathbf{L}_{(s, s')}$ applied to the probability field $\mathbf{u}$ should 
increase the probability of observing the current pose $\mathbf{p}$ given the current RGBD frame $\mathbf{\Omega}$ by reducing the 
variance of the camera tracking result. As such, global tracking variance is reduced by enforcing local consistency, improving the quality 
of the reconstruction.

A gradient based maximisation of the above posterior to yield an optimal deformation is a highly nonlinear optimisation problem. As such, it is suited 
to second order gradient based optimisation routines such as Gauss-Newton or Levenberg-Marquardt.
It should be noted that in our implementation the $P(\mathbf{\Omega} | \mathbf{p})$ term is assumed to be uniform in the case of an 
RGBD sensor being used, however for applications such as monocular SLAM this term may be replaced with a noise model when there is 
significant uncertainty about the given depth map at each frame.

The following proportionality to the distribution over deformations is made:
\begin{equation}
\begin{split}
P(\mathbf{L}_{(s, s')} | \mathbf{\Omega}, \mathbf{p}) \propto P(\mathbf{\Psi}_{s}(\mathbf{x}) | \mathbf{\Psi}_{s'}(\Lambda(\mathbf{x})))
\end{split}
\end{equation}
With the likelihood function taking the following form:
\begin{equation}
\begin{split}
P(\mathbf{\Psi}_{s'}(\Lambda(\mathbf{x}))) = \prod_{(s, s') \in \mathcal{S}} \frac{1}{\sqrt{2 \pi \sigma}} \exp{\frac{-(\mathbf{\Psi}_{s}(\mathbf{x}) - \mathbf{\Psi}_{s'}(\Lambda(\mathbf{x})))^2}{2\sigma^2}}
\end{split}
\end{equation}
Or alternatively:-
\begin{equation}
\begin{split}
\ln P(\mathbf{\Psi}_{s'}(\Lambda(\mathbf{x}))) = m\ln\frac{1}{\sqrt{2\pi}\sigma}\\
-\frac{1}{2\sigma^2} \sum_{(s, s') \in \mathcal{S}} \bigg( \mathbf{\Psi}_{s}(\mathbf{x}) - \mathbf{\Psi}_{s'}(G(\mathbf{x})) \bigg)^2
\end{split}
\end{equation}
Where $\mathbf{\Psi}(.)$ is a scalar valued SDF(Signed Distance Function), a dicretised field of $\mathbf{\Phi}$, as previously described. $\mathbf{x}$ is a point represented by a 3-vector and $\Lambda(.)$ is a transformation function taking the following form:
\begin{equation}
\begin{split}
\Lambda(\mathbf{x}) = \mathbf{R}(r_{1}, r_{2}, r_{3})\mathbf{x} + \mathbf{t}
\end{split}
\end{equation}
Where $\mathbf{R}(.)$ is a rotation matrix from the Special Orthogonal group $\mathbbm{SO}(3)$ paramaterised by the three 
Rodrigues Parameters\cite{Shuster1993} $r_{1}$, $r_{2}$ and $r_{3}$.

Note that the logarithmic form of the above likelihood is suitable to Nonlinear Least Squares optimisation, allowing the posterior of equation 4 
to be maximised in terms of the likelihood term of equation 4. To perform MLE(Maximum Likelihood Estimation) over this likelihood using 
an optimisation routine such as Levenberg Marquardt, the following gradients must be computed for the rotational component of the 
deformation:-
\begin{equation}
\begin{split}
\frac{\partial E}{\partial r_{n}} = \frac{\partial E}{\partial \mathbf{\Psi}} \frac{\partial \mathbf{\Psi}}{\partial \Lambda} \frac{\partial \Lambda}{\partial r_{n}} \text{for } n \in \{1,2,3\}
\end{split}
\end{equation}
Similarly for the translational component:-
\begin{equation}
\begin{split}
\frac{\partial E}{\partial \mathbf{t}_{d}} = \frac{\partial E}{\partial \mathbf{\Psi}} \frac{\partial \mathbf{\Psi}}{\partial \Lambda} \frac{\partial \Lambda}{\partial \mathbf{t}_{d}} \text{for } d \in  \{x,y,z\}
\end{split}
\end{equation}
where the gradient $\frac{\partial \mathbf{\Psi}}{\partial \Lambda}$ is found via finite differencing.

\section{Implicit Surface Deformations}
In the previous section, a model and estimation procedure was presented to find optimal transformations between the aforementioned subvolumes.
The overall object surface $\mathbf{\Phi}$ is implicitly deformed by a combining function $\mathbf{\zeta}(\mathbf{\Phi})$ over each of the subvolumes to which transformations have been applied.
As such the surface $\mathbf{\Phi}$ is given by the following:
\begin{equation}
\begin{split}
\mathbf{\Phi} = \int_{\chi \in X} \mathbf{\zeta}(\mathbf{\Phi}_{\chi}) d \mathbf{\Phi}_{\chi}
\end{split}
\end{equation}
where $X$ is the set of subvolumes contributing to the surface $\mathbf{\Phi}$.

Given this formulation...

\section{Volumetric Object Segmentation}
The final stage in the proposed object reconstruction pipeline is the segmentation of the object voxels from those that have had measurements fused 
from the background. This segmentation is formulated within a CRF framework, where each node in the CRF represents a set of neighbouring voxels in space, 
with connections being made between adjacent neighbourhoods. The process of segmentation can be posed as an energy minimisation problem over a cut in voxel space, 
such that a segmentation in 3D is obtained. The following energy function consists of the unary posterior probabilities over appearance accumulated during the fusion 
process for a region in space and an additional pairwise smoothing term representing the physical appearance similarity of the object region represented by the voxel 
neighbourhoods $\gamma$ and $\gamma^{'}$:
\begin{equation}
\begin{split}
E_{n} = \prod_{t=0}^{\infty} \prod_{\psi \in \mathbf{\Phi}_{n}} P(\psi \in \mathbf{\Psi} | \mathbf{\Omega}_{t}, \mathbf{p}_{t}) + P(\mathbb{E}[\mathbf{c}]_{\gamma} | \mathbb{E}[\mathbf{c}]_{\gamma^{'}})
\end{split}
\end{equation}
where $\mathbf{c}$ represents the set of colour measurements fused in to the voxels within a given neighbourhood, for all $N$ subvolumes.

\section{Implementation Notes}
The probabilities that are accumulated into the volume are generated from a Random Forest based appearance model using patch based features encompassing 
appearance and surface information, such as depth gradients, initialised prior to reconstruction by a user interaction in the first frame. There are two 
classes in the appearance model, one for the foreground object and one for the background, with the foreground object indicated by a bounding box on the 
first RGB frame.

An overview of the processing pipeline for the proposed system is outlined in Figure \ref{pipelineDiagram}.
\begin{figure*}[!t]
	\centering
	\includegraphics[scale=0.5]{pipeline.pdf}
	\caption{Object reconstruction pipeline.}
	\label{pipelineDiagram}
\end{figure*}

\section{Results}
To evaluate our system, we perform quantitative experiments on camera pose estimation accuracy, and qualitatively analyse the obtained reconstructions.
%To evaluate the performance of our system we perform experiments that draw comparisons in terms of quantitative pose estimation and qualitative reconstruction quality and efficacy.
Firstly, the pose estimation accuracy is evaluated via a well-established SLAM evaluation benchmark~\cite{sturm12iros}.
We like to point out that % with the primary difference being that with
in traditional dense SLAM systems~\cite{Prisacariu2014,Niessner2013,Newcombe2011} -- for which the benchmark is often employed -- the entire contents of the visible scene are used for pose estimation, whereas in our system we rely only on points belonging to the object's surface.
Whilst more challenging, this implicitly allows us to track the camera wrt. the object regardless of which of the two is subject to motion.
Then, qualitative comparisons are drawn between the reconstructions attained by our system, and those of the method described in \cite{Ren2013}.
We evaluate our system on multiple frame sequences depicting objects of different sizes. %; some with the object moving wrt. a fixed camera, others with the sensor in motion.% in the case of qualitative evaluation.

\vspace{-.7\baselineskip}

\subsection{Pose Estimation Quality}
In this section we present quantitative results of our systems' %ability to maintain tracking
robustness in estimating the camera motion, by performing tracking against the reconstruction of a single object only, instead of the whole scene.
The trajectories estimated by our system demonstrate low tracking drift. % and a robustness to loop closure events.
We perform such evaluation on two sequences of the RGB-D SLAM Dataset~\cite{sturm12iros} depicting static objects observed by a moving camera.
Tracking is performed using purely geometric clues, by matching the current depth frame with a rendering of the reconstructed object using a projective ICP tracking approach~\cite{Kahler2016}.
%, as such the evaluation is of camera pose estimation when tracking an object.
%By tracking the sensor pose against a subset of the observed scene 
%At this point it should be highlighted that our system is at a disadvantage when compared to dense SLAM systems that utilise the entire scene geometry for pose optimisation.

%In the following experiments, tracking is performed using only geometry cues from the rendered object models and the instantaneous depth frame.\\

The tracking accuracy is evaluated via the Absolute Trajectory Error (ATE) metric, as outlined in \cite{sturm12iros}, and is summarised in Table~\ref{ateTable}.

\begin{table}[!t]
	{
        \footnotesize
		\begin{center}
			\begin{tabular}{l@{\hskip 1cm} c}
				\emph{Sequence Name} & \emph{ATE (m)}\\
				\midrule
				\textsf{freiburg3\_cabinet} & 0.078 \\ %0.077903
				\textsf{freiburg3\_teddy}   & 0.031 \\ %0.030596
			\end{tabular}
		\end{center}
	}
	\caption{Absolute Trajectory Error (ATE) results (lower is better) achieved by our approach.}
	\label{ateTable}
\end{table}

\begin{figure}[!t]
	\centering
	\begin{tabular}{cc}
		\bmvaHangBox{\fbox{\includegraphics[width=0.25\textwidth]{results/rgbd_dataset_freiburg3_cabinet.png}}}&
		\bmvaHangBox{\fbox{\includegraphics[width=0.25\textwidth]{results/rgbd_dataset_freiburg3_teddy.png}}}
	\end{tabular}
	\caption{
		\textbf{(L)} Trajectory of the camera for the \textit{freiburg3\_cabinet} sequence versus ground truth.
		\textbf{(R)} Trajectory of the camera for the \textit{freiburg3\_teddy} sequence versus ground truth.
	}
\label{fig:tumTrajectories}
\end{figure}

At this point, it should be highlighted that our proposed system is at a disadvantage when compared to dense SLAM systems that utilise the entire scene geometry for pose optimisation, since we track the sensor pose against a subset of the observed scene.
Nevertheless, as shown by the results in Figure~\ref{fig:tumTrajectories}, our system is able to robustly estimate trajectories close to the ground truth whilst using only the objects' geometric appearance.
The cabinet reconstructed in the \textit{freiburg3\_cabinet} sequence is lacking in geometric features, as the object is mostly planar,
%It can be seen that there is a
and the small deficit in tracking quality is mostly due to this factor.
However, our system remains able to estimate a fairly accurate trajectory. 
In the \textit{freiburg3\_teddy} sequence we determine a trajectory very close to the ground truth.
Improvement over the accuracy in \textit{freiburg3\_cabinet} is due to the wider availability of geometrical features, such as curves in the teddy's body and head.

\subsection{Qualitative Reconstruction Quality}
In this section we present a qualitative comparison of our method vs. the approach by Ren et~al.~\cite{Ren2013} in the reconstruction of closed object models. %by relying on a variety of sequences and demonstrating efficacy over \cite{Ren2013} in this regard.
Each sequence is run through both systems; to evaluate the obtained results we regularly take snapshots of the reconstruction, in the case of our system, and the level set evolutions, in the case of \cite{Ren2013}.
Such snapshots are captured after each quarter of a sequence has been processed. %quarterly intervals of the systems run time through the sequence.

\begin{figure}[!t]
	\centering
	\begin{tabular}{cc}
		\bmvaHangBox{\fbox{\includegraphics[width=0.45\textwidth]{filmstrips/dino.png}}}&
		\bmvaHangBox{\fbox{\includegraphics[width=0.45\textwidth]{filmstrips/dino_s3d_large.png}}}
	\end{tabular}
	\caption{
        Quarterly interval snapshots of the Dinosaur Head reconstruction using \textbf{(L)} our method, and \textbf{(R)} the one proposed by Ren et~al.~\cite{Ren2013}.
	}
	\label{fig:dinoComparison}
\end{figure}

\begin{figure}[!t]
	\centering
	\begin{tabular}{ccc}
		\bmvaHangBox{\fbox{\includegraphics[width=0.2\textwidth]{screenshots/dino_colour_top.PNG}}}&
		\bmvaHangBox{\fbox{\includegraphics[width=0.2\textwidth]{screenshots/chair_colour_top.PNG}}}&
		\bmvaHangBox{\fbox{\includegraphics[width=0.2\textwidth]{screenshots/rock_colour_top.PNG}}}
	\end{tabular}
	\caption{
		Closed reconstructions of \textbf{(L)} a Dinosaur Head,
		\textbf{(M)} a Chair, and
		\textbf{(R)} a Rock.
	}
	\label{fig:top_shots}
	\vspace{-\baselineskip}
\end{figure}

As depicted in Figure~\ref{fig:dinoComparison}, our method is able to successfully reconstruct the Dinosaur Head, whereas the approach by Ren~et~al. fails to converge towards a feasible shape.
In addition, Figure~\ref{fig:top_shots} demonstrates that our system is able to generate consistent models (unaffected by camera tracking drift) for a variety of sequences containing several loop closures.
Failure of the competing method %of Ren et al
is also apparent for other sequences evaluated in this work, all presenting failure cases analogous to Figure~\ref{fig:dinoComparison}b. % to converge to a correct shape.
Such examples will be presented in the supplementary materials.
%The supplementary materials to this work demonstrate our efficacy vs that of Ren et al \cite{Ren2013}.\\

The object reconstructions depicted in Figure~\ref{fig:demo} have been obtained from sequences in which a camera was moved in a loop around each object in order to generate a closed model.

%As can be seen from the top down views of Figure \ref{fig:top_shots}, our system is capable of producing closed models, unaffected by tracking drift and loop closure events.

\section{Discussion}
As has been demonstrated in this paper, our system is efficacious in 3D object 
reconstruction. Our system is able to reconstruct closed object models on sequences for which an alternative, state of 
the art system \cite{Ren2013} fails to converge to any reasonable solution. In addition, we show robust odometry on an 
established SLAM benchmark, despite the difficulty of tracking only the objects surface vs the entire scene. Finally,
in spite of our use of ICP for pose estimation with ambiguities, our system is capable of robustly reconstructing a 
variety of objects.
 

%% if specified like this the section will be ommitted in review mode
\acknowledgements{
The authors wish to thank...}

\bibliographystyle{bibliography}
%%use following if all content of bibtex file should be shown
%\nocite{*}
\bibliography{template}
\end{document}
