Dense SLAM(Simultaneous Localisation and Mapping) has proven to be an effective paradigm for the reconstruction of scenes of moderate size,
with much research on the topic driven by the availablility of consumer grade depth sensing equipment. However, there is a heavy reliance on
descriptive geometry in the scene when there is a lack of texture. Less descriptive geometry leads to an increase in camera tracking error
and causes inconsistencies when a loop closure event occurs.

As object reconstruction can be seen as a smaller scale equivelant of the scene based dense reconstruction problem, it too is prone to the
tracking drift and loop closure problem, sometimes to a prohibitive level. Often it may be desirable to perform object reconstruction
in an interactive way, for example, as a component of a scene understanding system, or to procure training data for the object in question.
With a high level of interaction comes an exacerbation of the aformentioned shortcomings of dense SLAM, particulary due to the potential
for frequent, repetetive motion. This is the problem that is addressed in this work.

In this paper we present a probabilistic object reconstruction framework for in hand reconstruction of rigid objects based on object
appearances. The framework facilitates the correction of camera tracking drift by representing the object to be reconstructed as a
collection of overlapping subsegments such that deformations may be inferred to keep the subsegments aligned, resulting in a consistent
overall model. We utilise a volumetric representation for each of these object subsegments, as with many larger scale reconstruction
systems. Each voxel in the subsegments has additional appearance posterior information pertaining to the voxels membership of the object.
Over time, multiple volumes containing both surface and probabilistic appearance information are maintained and manipulated to yield a
robust and temporally consistent model.

In addition, to further increase the robustness of the object tracking, a novel tracking procedure is used, utilising appearance features
in image space. Finally, the optimum object shape is optimised for within a continuous max flow framework.