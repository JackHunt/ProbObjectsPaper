Dense SLAM (Simultaneous Localisation and Mapping) has proven to be an effective paradigm for the reconstruction of moderately-sized scenes,
with much recent research on the topic driven by the availability of inexpensive, consumer-grade depth sensing equipment \cite{Newcombe2011,Niessner2013,Prisacariu2014}. 
However, a prominent problem within the dense SLAM paradigm is that of tracking camera motion with respect to the scene. \STUART{It's arguable that it's half the point of SLAM, since SLAM involves localisation and mapping, and that's the localisation bit. I would word this slightly differently: it currently sounds a bit self-evident.} Two common approaches exploit geometry and texture 
respectively. \STUART{I know what you mean, but I wouldn't word it like this. It's common for tracking approaches to exploit geometry and/or texture cues, and various approaches exploit one or both to a greater or lesser extent. However, the way you've worded it sounds like you're talking about two specific approaches, neither of which you cite.} However, there are failure cases in both approaches when there is a lack of geometrical or texture based features to which the system must track against. \STUART{Or more precisely, trackers that rely purely on geometry cues tend to fail when there's no geometry, and trackers that rely purely on texture cues tend to fail when there's no texture. As worded, it's a little vague. I think fixing the previous point will help -- you need to make it clear that you're not talking about two specific approaches.}
Common issues range from tracking drift, often caused by ambiguity, to complete tracker failure. \STUART{It's not clear precisely what you mean by `ambiguity' here: this needs pinning down. Also, there are various causes of tracking drift beyond what I think you mean. As a separate question: when you talk about `common issues', what other issues did you have in mind?} In addition, the error incurred during loop closure events can be detrimental 
to tracking quality with the aforementioned approaches. \STUART{It's not obvious to me what you mean by `the aforementioned approaches' here: we're still talking in vague terms about trackers at this point. I think if we're going to point to things that are problematic in existing approaches, we need to make it clear which existing approaches we're talking about.} Failure cases for such approaches are only exacerbated when there is 
less data to track with/against, as is the case with the tracking of an object's pose w.r.t.\ the camera (the inverse of the camera tracking problem).

As object reconstruction can be seen as a smaller-scale equivalent of the scene-based dense reconstruction problem \STUART{Well, \emph{dense} object reconstruction can be.}, it too is prone to the
tracking drift and loop closure problem, sometimes to a prohibitive level. As previously highlighted, the tracking of an object's pose w.r.t.\ the camera is also affected 
by geometric or texture-based ambiguities, caused by a much more focused domain of the data available to the tracking algorithm. \STUART{Are the ambiguities caused by having less data available? Or are there just fewer unambiguous points available for tracking?} For example, when reconstructing an 
entire scene, all of the data in the current and previous frames is available. When tracking solely an object this is not the case. \STUART{Well, the data is arguably available, but you only want to use the bits corresponding to the object for tracking.}

In this paper, we present a probabilistic object reconstruction framework for in-hand reconstruction of rigid objects based on object
appearances. \STUART{Are we doing in-hand? Or are we doing things like chairs? Does our in-hand reconstruction work?} The framework facilitates the correction of camera tracking drift by representing the object to be reconstructed as a
collection of overlapping subsegments such that deformations may be inferred to keep the subsegments aligned, resulting in a consistent
overall model. We utilise a volumetric representation for each of these object subsegments, as with many larger-scale reconstruction
systems\cite{Kahler2016}. \STUART{Are there many systems that do what \cite{Kahler2016} does? Or is there just this one, and we're using it?} Each voxel in the subsegments has additional appearance posterior information pertaining to the voxel's membership of the object.
Over time, multiple volumes containing both surface and probabilistic appearance information are maintained and manipulated to yield a
robust and temporally-consistent model.

In addition, to further increase the robustness of the object tracking, a novel tracking procedure is used, utilising appearance features
in image space. Finally, we optimise for the optimum object shape within a max-flow framework.

We demonstrate an improvement in object reconstruction quality by performing online transformations to the aforementioned subsegments to counteract inconsistencies 
caused by drift in the object pose tracking process. \STUART{An improvement in quality with respect to what?}
We perform (...) experiments and demonstrate a significant qualitative improvement. In addition we demonstrate that the proposed system is quantitatively better 
with regards to (...) by performing (...) experiments.

TO-DO:- Decide upon and perform experiments and expand on this here.